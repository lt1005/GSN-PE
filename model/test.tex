\documentclass{article}
\usepackage{amsmath}
\usepackage{amsfonts}
\usepackage{amssymb}
\usepackage{graphicx}
\usepackage{geometry}
\geometry{a4paper, margin=1in}

% --- 中文支持设置 (请使用 XeLaTeX 编译器编译) ---
\usepackage{xeCJK} % 加载 xeCJK 宏包以支持中文
\setCJKmainfont{SimSun} % 设置主中文字体。请确保您的系统安装了此字体。
                        % 如果没有,请替换为其他已安装的字体,例如 "Microsoft YaHei" 或 "KaiTi"
% 如果您的 LaTeX 发行版较新,也可以尝试直接使用 ctex 宏包,它通常会自动配置字体:
% \usepackage{ctex}
% --------------------------------------------------

% --- TikZ 绘图宏包及库 ---
\usepackage{tikz}
\usetikzlibrary{positioning, arrows.meta, backgrounds}

% --- 自定义颜色定义 (替换掉原先不识别的十六进制格式) ---
\usepackage{xcolor}
\definecolor{reasoningcolor}{RGB}{180,199,231} % 对应原先的 ##B4C7E7 (浅蓝色)
\definecolor{triggercolor}{RGB}{178,206,231}   % 对应原先的 blue!20
\definecolor{calculatorcolor}{RGB}{198,225,248} % 对应原先的 green!20
\definecolor{updatercolor}{RGB}{255,204,153}   % 对应原先的 orange!20
\definecolor{entitycolor}{RGB}{180,199,231} % 对应原先的 ##B4C7E7,用于实体节点

% --- TikZ 样式定义 (根据您的错误信息推断可能需要定义这些样式) ---
\tikzset{
    box/.style={
        rectangle,
        rounded corners,
        draw=black,
        fill=white,
        thick,
        text width=4.5cm,
        minimum height=1.5cm,
        align=center
    },
    process_box/.style={ % 用于流程中的步骤
        rectangle,
        draw,
        fill=white,
        thick,
        text width=3.5cm,
        minimum height=1cm,
        align=center
    },
    entity/.style={ % 用于实体节点
        rectangle,
        rounded corners,
        draw=black,
        fill=entitycolor, % 使用自定义的实体颜色
        thick,
        minimum width=2.5cm,
        minimum height=0.8cm,
        align=center
    },
    arrow/.style={
        -Stealth, % 箭头样式
        thick
    },
    comment/.style={
        text width=3cm,
        align=left,
        font=\small,
        text=gray
    }
}

\title{Anchor Subgraph Expansion 与结构嵌入对齐技术方案(完整版)}
\author{}
\date{}

\begin{document}

\maketitle

\section{目标}
\begin{itemize}
    \item 针对规则 $XX$ 的谓词集合,从知识图谱中采样最小支持子图 $G_t$,确保结构上与规则匹配;
    \item 通过结构嵌入 $f_{\text{struct}}$ 实现规则与子图的可对齐和判定;
    \item 保证边扩展过程高效且准确,避免枚举爆炸。
\end{itemize}

\section{预备定义}
\begin{itemize}
    \item \textbf{规则前件谓词集合}: $X = \{ p_1, p_2, ..., p_m \}$,构造规则图 $G_X$(无实体,仅谓词节点及边);
    \item \textbf{结构 motif 模板集}: $\mathcal{M} = \{ m_1, m_2, ..., m_d \}$,如路径、三角、星型等结构模板;
    \item \textbf{结构嵌入函数}:对任意图 $G$,通过 motif 计数线性映射,得到结构向量:
    $$f_{\text{struct}}(G) = W \cdot \text{motif\_counts}(G, \mathcal{M}) \in \mathbb{R}^d$$
\end{itemize}

\section{规则结构编码}
\begin{itemize}
    \item 计算规则图 $G_X$ 在 motif 集 $\mathcal{M}$ 上的计数:
    $$\text{motif\_counts}(G_X, \mathcal{M}) = [c_1, c_2, ..., c_d]$$
    \item 得到规则结构嵌入:
    $$f_{\text{struct}}(X) := f_{\text{struct}}(G_X)$$
    \item 该嵌入作为结构对齐的“锚点”。
\end{itemize}

\section{Anchor Predicate Expansion — 子图采样流程}
\subsection{初始化}
\begin{itemize}
    \item 遍历知识图谱中满足任一谓词 $p_i \in X$ 的三元组 $(h, p_i, t)$,将其作为扩展起点;
    \item 初始子图 $G_0 = \{ (h, p_i, t) \}$,计算结构嵌入 $f_{\text{struct}}(G_0)$。
\end{itemize}

\subsection{递归边扩展}
对于当前子图 $G_t$,候选扩展边集合为与 $G_t$ 节点相邻的知识图谱边,考虑候选边 $e$:
\begin{itemize}
    \item 计算扩展后子图 $G_{t+1} = G_t \cup \{ e \}$ 的结构嵌入:$f_{\text{struct}}(G_{t+1})$
    \item 计算扩展相似度增益(L2距离):
    $$\Delta = \| f_{\text{struct}}(G_t) - f_{\text{struct}}(X) \|_2 - \| f_{\text{struct}}(G_{t+1}) - f_{\text{struct}}(X) \|_2$$
    \item \textbf{扩展准则}:
    \begin{itemize}
        \item 若 $\Delta \geq \tau$(预设阈值),表示扩展使子图结构更接近规则,接受该边扩展;
        \item 否则,舍弃该边。
    \end{itemize}
\end{itemize}

\subsection{终止条件}
\begin{itemize}
    \item 子图包含规则全部谓词 $X$(即谓词集合 $X \subseteq \text{predicates}(G_t)$);
    \item 结构嵌入距离满足:
    $$\| f_{\text{struct}}(G_t) - f_{\text{struct}}(X) \|_2 \leq \epsilon$$
    \item 达到预设最大子图大小或深度。
\end{itemize}

\section{负样本采样}
\begin{itemize}
    \item 对规则谓词集 $X$ 随机扰动,替换、添加或删除谓词,得到 $X' \neq X$;
    \item 基于 $X'$ 同样执行 Anchor Predicate Expansion 采样子图,作为负样本。
\end{itemize}

\section{对齐与训练目标}
\begin{itemize}
    \item 通过结构嵌入 $f_{\text{struct}}$ 和联合语义嵌入 $f$,训练规则与子图嵌入对齐模型;
    \item 损失函数鼓励正样本距离最小,负样本距离较大。
\end{itemize}

\section{总结}
\begin{itemize}
    \item 结构嵌入 $f_{\text{struct}}$ 统一规则和子图的结构表示空间,实现可对齐;
    \item 扩展边的判定基于结构相似度增益,防止盲目扩张,保证子图最小且高质量;
    \item 最终得到与规则结构高度匹配的实例子图,促进规则推理和知识发现。
\end{itemize}

% --- 示例 TikZ 图形 (根据您提供的错误信息中的中文节点推断) ---
% 如果您需要绘制具体的流程图,请提供更详细的图结构描述。
% 这里仅仅是一个基于错误信息推断的简单示例。
\newpage % 另起一页放置图形,保持文档整洁

\section*{示例流程图}
\begin{figure}[htbp]
    \centering
    \begin{tikzpicture}[node distance=1.5cm and 2cm]
        % 主流程框
        \node[box, fill=reasoningcolor] (reasoning) {关系推理层};

        % 内部组件
        \node[process_box, fill=triggercolor] (trigger) at (reasoning.160) {规则触发器};
        \node[process_box, below=of trigger] (c3) {规则C3匹配};
        \node[process_box, below=of c3] (verify) {输入实例校验};

        \node[process_box, fill=calculatorcolor] (calculator) at (reasoning.200) {动态计算器};
        \node[process_box, below=of calculator] (result) {风险指数};

        \node[process_box, fill=updatercolor] (updater) at (reasoning.240) {知识更新器};
        \node[process_box, below=of updater] (add) {添加保护剂节点};
        \node[process_box, below=of add] (modify) {修改剂量标签};

        % 连接线
        \draw[arrow] (trigger) -- node[left] {触发} (c3);
        \draw[arrow] (c3) -- node[left] {校验结果} (verify);
        \draw[arrow] (reasoning.south west) |- (calculator); % 示例连接

        % 实体节点示例
        \node[entity] (patient) at (8, 0) {患者P001};
        \node[entity, below=of patient] (drug) {药物A};
        \node[entity, below=of drug] (disease) {疾病B};

        % 实体节点之间的关系示例
        \draw[arrow] (patient) -- node[above, sloped] {患有} (disease);
        \draw[arrow] (patient) -- node[above, sloped] {服用} (drug);

    \end{tikzpicture}
    \caption{关系推理层流程示例}
    \label{fig:reasoning_flow}
\end{figure}

\end{document}